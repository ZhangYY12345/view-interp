%%% template.tex
%%%
%%% This LaTeX source document can be used as the basis for your technical
%%% paper or abstract. Intentionally stripped of annotation, the parameters
%%% and commands should be adjusted for your particular paper - title, 
%%% author, article DOI, etc.
%%% The accompanying ``template.annotated.tex'' provides copious annotation
%%% for the commands and parameters found in the source document. (The code
%%% is identical in ``template.tex'' and ``template.annotated.tex.'')

\documentclass[conference]{acmsiggraph}

\TOGonlineid{45678}
\TOGvolume{0}
\TOGnumber{0}
\TOGarticleDOI{1111111.2222222}
\TOGprojectURL{}
\TOGvideoURL{}
\TOGdataURL{}
\TOGcodeURL{}

\title{Planar Depth Reconstruction for Image Based Rendering}

\author{Puneet Lall\thanks{e-mail:pkl5rc@virginia.edu}\\University of Virginia}
\pdfauthor{Puneet Lall}

%% \keywords{radiosity, global illumination, constant time}

\begin{document}

%% \teaser{
%%   \includegraphics[height=1.5in]{images/sampleteaser}
%%   \caption{Spring Training 2009, Peoria, AZ.}
%% }

\maketitle

\begin{abstract}

% Citations can be done this way~\cite{Jobs95} or this more concise 
% way~\shortcite{Jobs95}, depending upon the application.

    TODO Abstract goes here~\cite{furukawa2010accurate}.

\end{abstract}

% \begin{CRcatlist}
  % \CRcat{I.3.3}{Computer Graphics}{Three-Dimensional Graphics and Realism}{Display Algorithms}
  % \CRcat{I.3.7}{Computer Graphics}{Three-Dimensional Graphics and Realism}{Radiosity};
% \end{CRcatlist}

% \keywordlist

%% Use this only if you're preparing a technical paper to be published in the 
%% ACM 'Transactions on Graphics' journal.

% \TOGlinkslist

%% Required for all content. 

\copyrightspace

\section{Introduction}

Image-based rendering provides a practical solution to the task of creating
3D models for rendering of complex scenes, particularly in the case of
novice users who wish to quickly create a visualization of a real-world object.
Rather than requiring users to explicitly create a three-dimensional model using
sophisticated modeling software, an image-based rendering system may, for example, 
enable users to render a scene constructed implicitly from a sequence of
digital photographs taken by the user.  Thus, such a system can enable
non-expert users to relatively-quickly capture and share a representation of
a three-dimensional scene.

This paper presents a 
 
Shum and Kang \shortcite{shum2000review} presented a review of various image based
rendering techniques and showed that they can fall onto a spectrum defined
by the representation of their geometric proxy.  While some systems,
such as the Lumigraph \cite{gortler1996lumigraph} construct 

\section{Approach}

Introduction to the approach.  In the following equation:

\begin{equation}
 \sum_{j=1}^{z} j = \frac{z(z+1)}{2}
\end{equation}

\begin{eqnarray}
x & \ll & y_{1} + \cdots + y_{n} \\
  & \leq & z
\end{eqnarray}

More stuff!

\section{Structure from Motion}

This part outlines the structure from motion approach.
\begin{figure}[ht]
  \centering
  \includegraphics[width=1.5in]{images/samplefigure}
  \caption{Sample illustration.}
\end{figure}
More explanation.  Blah blah blah

Lorem ipsum dolor sit amet, consectetur adipisicing elit, sed do
eiusmod tempor incididunt ut labore et dolore magna aliqua. Ut enim ad
minim veniam, quis nostrud exercitation ullamco laboris nisi ut
aliquip ex ea commodo consequat. Duis aute irure dolor in
reprehenderit in voluptate velit esse cillum dolore eu fugiat nulla
pariatur. Excepteur sint occaecat cupidatat non proident, sunt in
culpa qui officia deserunt mollit anim id est laborum.

Lorem ipsum dolor sit amet, consectetur adipisicing elit, sed do
eiusmod tempor incididunt ut labore et dolore magna aliqua. Ut enim ad
minim veniam, quis nostrud exercitation ullamco laboris nisi ut
aliquip ex ea commodo consequat. Duis aute irure dolor in
reprehenderit in voluptate velit esse cillum dolore eu fugiat nulla
pariatur. Excepteur sint occaecat cupidatat non proident, sunt in
culpa qui officia deserunt mollit anim id est laborum.

\section{Planar depth reconstruction}

This is how planar depth reconstruction works!

\section{Results}

Talk about results.

Also talk about limitations.

\section{Conclusion}

In this paper, I have described the implementation of a system for
reconstruction of a planar representation of dense depth maps from multiple images
with an application to image-based rendering.

\bibliographystyle{acmsiggraph}
\bibliography{template}
\end{document}
